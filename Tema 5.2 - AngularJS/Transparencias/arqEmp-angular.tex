\documentclass[a4paper,slidestop,xcolor=pst,blue]{beamer}

\input{slidesHeader.tex}

\title[Angular]{Desarrollo de Interfaces Web con Angular}

\author[P. S{\'a}nchez]{\alert{Pablo S{\'a}nchez}}

\institute[IIE]{
		   Dpto. Ingenier{\'i}a Inform{\'a}tica y Electr{\'o}nica \\
		   Universidad de Cantabria \\
		   Santander (Cantabria, Espa{\~n}a) \\
		   \texttt{p.sanchez@unican.es}
}

\date{}

\begin{document}

\begin{frame}[c]
	\titlepage
	\begin{columns}
		\column{0.50\linewidth}
			\centering
    		\includegraphics[width=.28\textwidth,keepaspectratio=true]{images/istr.eps}
		\column{0.50\linewidth}
			\centering
			\includegraphics[width=.25\textwidth,keepaspectratio=true]{images/uc.eps}
	\end{columns}
\end{frame}

\begin{frame}[c]
    \frametitle{\alert{Advertencia}}
    \begin{center}
        Todo el material contenido en este documento no constituye en modo alguno una obra de referencia o apuntes oficiales mediante el cual se puedan preparar las pruebas evaluables necesarias para superar la asignatura. \ \\
        \ \\
        Este documento contiene exclusivamente una serie de diapositivas cuyo objetivo es servir de complemento visual a las actividades realizadas en el aula para la transmisi{\'o}n del contenido sobre el cual versar{\'a}n las mencionadas pruebas evaluables.  \ \\
        \ \\
        Dicho de forma m{\'a}s clara, \alert{estas transparencias no son apuntes y su objetivo no es servir para que el alumno pueda preparar la asignatura.}
    \end{center}
\end{frame}

\section{Introducción}

\subsection{Objetivos y Bibliografía}

\begin{frame}[c]
    \frametitle{Objetivos del Tema}
    \begin{enumerate}[<+->]
         \item Conocer los diferentes elementos que integran Angular.
         \item Comprender el concepto de componente Angular.
         \item Saber crear un router en Angular.
         \item Saber crear páginas dinámicas en Angular.
         \item Saber especificar servicios en Angular.
         \item Saber utilizar el patrón MVVM en Angular.
         \item Saber realizar llamadas AJAX en Angular.
         \item Saber utilizar filtros en Angular.
    \end{enumerate}
\end{frame}

\begin{frame}[c]
    \frametitle{Bibliografía}
    \begin{thebibliography}{1}

        \bibitem[Google, 2019]{angular2019}
        Google (2019).
        \newblock {\em {Angular Docs}}.
        \newblock https://angular.io/docs

    \end{thebibliography}
\end{frame}

\section{Angular CLI}

\begin{frame}[c]
    \frametitle{Comandos Angular CLI}
    \begin{tabular}[ll]
         \textbf{Serve}    & Inicializa un servidor web. \\
         %% Invocar desde fuera de src
         \textbf{Generate} & Crear un componente   
    \end{enumerate}
\end{frame}


\section{Componente Gráfico}

\section{Eventos}

%% (click)

\section{LLamadas AJAX}

%% Concepto de Observable (Optional)
%% tap: realizar una acción por cada elemento de un observable y retorna el observable
%% http get, put
%% http options

\section{Sumario y Conclusiones}

\begin{frame}[c]
    \frametitle{Sumario}
    %% Definición
\end{frame}

\begin{frame}[c]
    \frametitle{Conclusiones}
    %% Definición
\end{frame}


\end{document}
