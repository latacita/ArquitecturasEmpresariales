\documentclass[a4paper,slidestop,xcolor=pst,blue]{beamer}

\input{slidesHeader.tex}

\usepackage{pifont}
\newcommand{\cmark}{\ding{51}}
\newcommand{\xmark}{\ding{55}}

\title[Spring Data + JPA]{Capa de Persistencia \\ \ \\ Spring Data + Java Persistence API (JPA)}

\author[P. S{\'a}nchez]{\alert{Pablo S{\'a}nchez}}

\institute[IIE]{
		   Dpto. Ingenier{\'i}a Inform{\'a}tica y Electr{\'o}nica \\
		   Universidad de Cantabria \\
		   Santander (Cantabria, Espa{\~n}a) \\
		   \texttt{p.sanchez@unican.es}
}

\date{}

\begin{document}

\begin{frame}[c]
	\titlepage
	\begin{columns}
		\column{0.50\linewidth}
			\centering
    		\includegraphics[width=.28\textwidth,keepaspectratio=true]{images/istr.eps}
		\column{0.50\linewidth}
			\centering
			\includegraphics[width=.25\textwidth,keepaspectratio=true]{images/uc.eps}
	\end{columns}
\end{frame}

\begin{frame}[c]
    \frametitle{\alert{Advertencia}}
    \begin{center}
        Todo el material contenido en este documento no constituye en modo alguno una obra de referencia o apuntes oficiales mediante el cual se puedan preparar las pruebas evaluables necesarias para superar la asignatura. \ \\
        \ \\
        Este documento contiene exclusivamente una serie de diapositivas cuyo objetivo es servir de complemento visual a las actividades realizadas en el aula para la transmisi{\'o}n del contenido sobre el cual versar{\'a}n las mencionadas pruebas evaluables.  \ \\
        \ \\
        Dicho de forma m{\'a}s clara, \alert{estas transparencias no son apuntes y su objetivo no es servir para que el alumno pueda preparar la asignatura.}
    \end{center}
\end{frame}

\section{Introducción}



\section{Sumario}

\begin{frame}[c]
    \frametitle{¿Qué tengo que saber de todo ésto?}
    \begin{enumerate}[<+->]
        \item Ser capaz de utilizar anotaciones JPA para especificar correspondencias objeto-relacional.
        \item Comprender cómo funcionan los repositorios \emph{Spring}.
    \end{enumerate}
\end{frame}

\end{document}
