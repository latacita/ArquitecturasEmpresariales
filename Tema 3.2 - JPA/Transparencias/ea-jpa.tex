\documentclass[a4paper,slidestop,xcolor=pst,blue]{beamer}

\input{slidesHeader.tex}


\usepackage{tabularx}
\newcommand{\ann}[1]{\color{blue}{\texttt{#1}}}

\title[JPA + Spring Data]{Capa de Persistencia \\ \ \\ Java Persistence API (JPA) + Spring Data}

\author[P. S{\'a}nchez]{\alert{Pablo S{\'a}nchez}}

\institute[IIE]{
		   Dpto. Ingenier{\'i}a Inform{\'a}tica y Electr{\'o}nica \\
		   Universidad de Cantabria \\
		   Santander (Cantabria, Espa{\~n}a) \\
		   \texttt{p.sanchez@unican.es}
}

\date{}

\begin{document}

\begin{frame}[c]
	\titlepage
	\begin{columns}
		\column{0.50\linewidth}
			\centering
    		\includegraphics[width=.28\textwidth,keepaspectratio=true]{images/istr.eps}
		\column{0.50\linewidth}
			\centering
			\includegraphics[width=.25\textwidth,keepaspectratio=true]{images/uc.eps}
	\end{columns}
\end{frame}

\begin{frame}[c]
    \frametitle{\alert{Advertencia}}
    \begin{center}
        Todo el material contenido en este documento no constituye en modo alguno una obra de referencia o apuntes oficiales mediante el cual se puedan preparar las pruebas evaluables necesarias para superar la asignatura. \ \\
        \ \\
        Este documento contiene exclusivamente una serie de diapositivas cuyo objetivo es servir de complemento visual a las actividades realizadas en el aula para la transmisi{\'o}n del contenido sobre el cual versar{\'a}n las mencionadas pruebas evaluables.  \ \\
        \ \\
        Dicho de forma m{\'a}s clara, \alert{estas transparencias no son apuntes y su objetivo no es servir para que el alumno pueda preparar la asignatura.}
    \end{center}
\end{frame}

\section{Introducción}

\begin{frame}
    \frametitle{Puentes de Persistencia de Objetos}
    \rput[lt](0,0){
        \includegraphics[width=\linewidth]{images/intro/orm02.eps}
    }
\end{frame}

\begin{frame}[c]
    \frametitle{JPA + Spring Data}
    \begin{block}{JPA}
        \emph{Java Persistence API (JPA)} es una especificación de referencia de un puente de persistencia estándar en Java.
        \begin{enumerate}
            \item<2-> Proporciona mecanismos para especificar el \emph{metadata mapping}.
            \item<3-> Especifica la interfaz y funcionamiento que deben tener una serie de elementos para realizar el acceso a datos (e.g. \emph{EntityManager}).
            \item<4-> Precisa de un ORM que la implemente (e.g. Hibernate).
        \end{enumerate}
    \end{block}
    \uncover<5->{
        \begin{block}{Spring Data}
            Framework que proporciona facilidades para la definición de repositorios compatible con diversas tecnologías de persistencia, entre ellas JPA.
        \end{block}
    }
\end{frame}

\begin{frame}[c]
    \frametitle{Objetivos del Tema}
    \begin{enumerate}[<+->]
         \item Ser capaz de transformar un conjunto de POJOs en un esquema relacional usando anotaciones JPA.
         \item Ser capaz crear repositorios JPA y \emph{Spring}.
         \item Ser capaz de utilizar repositorios Spring para interactuar con la capa de persistencia.
    \end{enumerate}
\end{frame}

\begin{frame}[c]
    \frametitle{Bibliografía}
    \begin{thebibliography}{1}

        \bibitem[Bauer, 2015]{Bauer2015}
        Bauer, C., King. G. y Gregory G. (2015).
        \newblock {\em {Java Persistence with Hibernate}}. 2ª Ed.
        \newblock Manning

        %% Spring Data

    \end{thebibliography}
\end{frame}

\section{Transformación Estructural}

\subsection{Metadata Mapping}

\begin{frame}[c]
    \frametitle{Metadata mapping}
    \begin{enumerate}[<+->]
         \item<1-> Se realiza mediante dos alternativas: anotaciones Java o basada en fichero XML.
         \item<2-> Anotaciones Java
             \begin{enumerate}
                \item<3-> Contaminan los POJOs, haciéndolos menos reutilizables y legibles.
                \item<4-> No requieren la actualización de ficheros externos cuando el modelo de dominio evoluciona.
             \end{enumerate}
         \item<5-> Ficheros XML
             \begin{enumerate}
                \item<6-> Permiten tener POJOs más limpios y reutilizables.
                \item<7-> Requieren ser actualizados cuando el modelo de dominio cambia,
             \end{enumerate}
    \end{enumerate}
\end{frame}

\subsection{Entidades y Value Objects}

\begin{frame}[c]
    \frametitle{Entidades y Value Objects}
    \begin{tabular}{lp{0.70\linewidth}}
        \ann{@Entity}     & La clase es una entidad y se transformará a una tabla. \\
        \ann{@Embeddable} & La clase es un \emph{value object} y será incrustada en otra.
    \end{tabular}
\end{frame}

\subsection{Referencias y Colecciones}

\begin{frame}[c]
    \frametitle{Transformación de Propiedades}
    \begin{enumerate}[<+->]
        \item<1-> Basada en campos.
            \begin{itemize}
                \item<2-> Se anotan los atributos a transformar.
                \item<3-> La carga y el almacenamiento se hace basado en reflexión.
            \end{itemize}
        \item<3-> Basada en \emph{getters}.
            \begin{itemize}
                \item<4-> Se anotan los \emph{getters} de los atributos a transformar.
                \item<5-> La carga y almacenamiento se hace basada en \emph{getters} y \emph{setters}.
                \item<6-> Require de \emph{getters} y \emph{setters} públicos para todos los atributos.
            \end{itemize}
    \end{enumerate}
\end{frame}

\begin{frame}[c]
    \frametitle{Transformación de Asociaciones}
    \begin{tabular}{ll}
        \ann{@OneToOne}   & Una referencia simple pertenece a una asociación 1-1. \\
        \ann{@ManyToOne}  & Una referencia simple pertenece a una asociación 1-n. \\
        \ann{@Embedded}   & Una referencia simple será incrustada. \\
        \ann{@OneToMany}  & Una colección pertenece a una asociación 1-n. \\
        \ann{@ManyToMany} & Una colección pertenece a una asociación 1-1. \\
        \ann{@Element}    &  \\
        \multicolumn{1}{r}{\ann{Collection}} & Un colección es de \emph{value objects}. \\
        \ann{@Lob}        & Una referencia se tratará como un LOB. \\
        \ann{@MapKey}     & Permite controlar mejor la transformación de un mapa. \\
    \end{tabular}
\end{frame}

\begin{frame}[c]
    \frametitle{Transformación de Asociaciones - Detalles}
    \begin{tabular}{lp{0.70\linewidth}}
        \ann{cascade}  & Define qué operaciones deben propagarse a la clase referenciada. \\
        \ann{mappedBy} & En una asociación bidireccional, especifica el extremo que hará de clave externo. \\
        \ann{fetch}    & Indica si la referencia se carga bajo demanda \texttt{LAZY} o no (\texttt{EAGER}). \\
    \end{tabular}
\end{frame}

\subsection{Identity Field}

\begin{frame}[c]
    \frametitle{Identity Field}
    \begin{tabular}{lp{0.70\linewidth}}
        \ann{@Id} & El atributo marcado será la clave primaria. \\ 
        \ann{@GeneratedValue} & El atributo será una clave autogenerada. \\
        \multicolumn{1}{r}{\texttt{strategy}} & 
                Especifica la estrategia de generación. \\
        \ann{@IdClass}    & \\
        \ann{@EmbeededId} & Permiten crear claves primarias compuestas.\\
    \end{tabular}
\end{frame}

\begin{frame}[c]
    \frametitle{Identity Field - Estrategias de Generación}
    \begin{tabular}{lp{0.70\linewidth}}
        \ann{AUTO}     & Se deja a decisión del ORM la elección. \\
        \ann{IDENTITY} & Generada por el gestor de pesistencia. \\
        \ann{SEQUENCE} & Usa una \emph{secuencia} del gestor de persistencia. \\
        \ann{TABLE}    & Generada por el propio ORM.
    \end{tabular}
\end{frame}

\subsection{Herencias}

%\begin{frame}[c]
%    \frametitle{Transformación de Herencias}
%    \begin{description}
%        \item[@@Inheritance(strategy=)]
%        \item[@DiscriminatorValue]
%    \end{description}
%\end{frame}
%
%\begin{frame}[c]
%    \frametitle{Personalización de Elementos}
%    \begin{description}
%        \item[@Table]
%        \item[@Column]
%        \item[@JoinColumn]
%        \item
%    \end{description}
%\end{frame}

%\section{Spring Repositories}
%
%\section{Sumario}
%
%\begin{frame}[c]
%    \frametitle{¿Qué tengo que saber de todo ésto?}
%    \begin{enumerate}[<+->]
%        \item Ser capaz de utilizar anotaciones JPA para especificar correspondencias objeto-relacional.
%        \item Comprender cómo funcionan los repositorios \emph{Spring}.
%    \end{enumerate}
%\end{frame}

\end{document}
