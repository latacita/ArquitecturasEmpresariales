\documentclass[a4paper,t,xcolor=pst,colortheme]{beamer}

\input{slidesHeader.tex}

\title[Aplicaciones Empresariales]{Tecnologías para el Desarrollo de Aplicaciones Empresariales sobre Internet}

\author[Pablo Sánchez]{\alert{Pablo Sánchez}}

\institute[I2E]{
		   Dpto. Ingenier{\'i}a Inform{\'a}tica y Electr{\'o}nica \\
		   Universidad de Cantabria \\
		   Santander (Cantabria, España) \\
		   p.sanchez@unican.es
}

\date{}

\begin{document}

\begin{frame}[c]
	\titlepage
	\begin{columns}
		\column{0.50\linewidth}
			\centering
    		\includegraphics[width=.28\textwidth,keepaspectratio=true]{images/istr.eps}
		\column{0.50\linewidth}
			\centering
			\includegraphics[width=.25\textwidth,keepaspectratio=true]{images/uc.eps}
	\end{columns}
\end{frame}

\section{Índice}

\section{Objetivos y Temario}

\subsection{Objetivos}

\begin{frame}[c]
   \frametitle{Objetivos de la Asignatura}
   \begin{enumerate}[<+->]
        \item Comprender el concepto de aplicación empresarial.
        \item Comprender cómo se divide una aplicación empresarial en capas.
        \item Saber aplicar patrones para el diseño de una capa de negocio.
        \item Saber aplicar los principios de \emph{Domain-Driven Design}.
        \item Comprender el funcionamiento de los patrones de persistencia.
        \item Saber utilizar herramientas para la generación de puentes objeto-relacional.
        \item Saber aplicar patrones para el diseño de una capa de servicio.
        \item Saber especificar e implementar una capa de servicio REST.
        \item Saber aplicar patrones para el diseño de una capa de presentación.
        \item Ser capaz de utilizar los servicios, como control de accesos, proporcionados por los servidores de aplicaciones.
	\end{enumerate}
\end{frame}

\begin{frame}[c]
   \frametitle{Resultados de Aprendizaje}
   \begin{enumerate}[<+->]
        \item Elaborar modelos de dominio utilizando \emph{POJOs}.
        \item Clasificar elementos de dominio conforme a la metodología \emph{Domain-Driven Design}.
        \item Anotar un modelo de dominio para persistirlo utilizando \emph{JPA}.
        \item Usar \emph{Spring} para la definición de repositorios de acceso a datos.
        \item Especificar una interfaz \emph{REST} para la manipulación de  un modelo de dominio.
        \item Serializar objetos en formato \emph{JSON} usando anotaciones.
        \item Implementar una controlador \emph{REST} usando \emph{Spring}.
        \item Implementar una capa de servicio usando \emph{Spring}.
        \item Implementar una interfaz \emph{SPA} utilizando \emph{Angular}.
        \item Ser capaz de utilizar \emph{tokens de autenticación}.
        \item Utilizar los servicios de \emph{Spring} para el \emph{control de accesos}.
	\end{enumerate}
\end{frame}

\subsection{Temario}

\begin{frame}[c]
	\frametitle{Temario}
	\begin{enumerate}
		\item<1-> Fundamentos de las Arquitecturas de Sistemas Empresariales.
		\item<2-> Capa de Negocio: Entidades de Dominio.
		\item<3-> Capa de Persistencia.
		\item<4-> Capa de Negocio: Servicios.
		\item<5-> Capa de Presentación.
        \item<6-> Seguridad: Control de Accesos.
	\end{enumerate}
\end{frame}

\section{Metodología}

\subsection{Actividades}

\begin{frame}[t]
	\frametitle{Clases en Aula}
	\begin{block}{Objetivo}
		Entender los conocimientos teóricos que constituyen la base de las habilidades y destrezas a adquirir al final de la asignatura.
	\end{block}
	\begin{itemize}
		\item<2-> Clases magistrales, fundamentalmente de pizarra.
		\item<3-> Sin conocimiento teórico es imposible alcanzar las habilidades prácticas.
	\end{itemize}
\end{frame}

\begin{frame}
	\frametitle{Clases en Laboratorio}
	\begin{block}{Objetivo}
		Aplicar los conceptos teóricos aprendidos en las clases de aula al desarrollo de un sistema software real de mediana escala, con el objetivo de desarrollar las competencias procedimentales y actitudinales propias de un Ingeniero.
	\end{block}
	\begin{itemize}
		\item<2-> Desarrollo de una aplicación empresarial de tamaño y complejidad moderada.
		\item<3-> El desarrollo estará dividido en varias etapas, al final de cada una de las cuales se realizará su correspondiente entrega.
	\end{itemize}
\end{frame}

\subsection{Plataforma}

\begin{frame}[c]
	\frametitle{Plataforma de Trabajo}
	\begin{itemize}
		\item<1-> Las plataformas de trabajo de la asignatura son \emph{Moodle} y \emph{Discord}.
		\item<2-> Todas las notificaciones oficiales se harán a través de Discord.
		\item<3-> Es obligación del alumno estar atento a las posibles notificaciones y avisos que se realicen a través de Discord.
	\end{itemize}
\end{frame}

\subsection{Horarios}

\begin{frame}[c]
	\frametitle{Horario de Clases}
	\begin{small}
		\only<1|handout:0>{
			\begin{center}
				\textbf{Tercer Periodo} \\ \ \\
				\begin{tabular}{||l|c|c|c|c|c||}
					\hline \hline
					& Lunes & Martes  & Miércoles & Jueves  & Viernes \\ \hline \hline
					15:30 - 16:30  &       & Aula 12 &           &         &         \\ \hline
					16:30 - 17:30  &       & Aula 12 &           &         &         \\ \hline
					17:30 - 18:30  &       &         &           &         & Aula 12 \\ \hline
					18:30 - 19:30  &       &         &           &         & Aula 12 \\ \hline
					\hline
				\end{tabular}
			\end{center}
		}
		\only<2-|handout:0>{
			\begin{center}
				\textbf{Cuarto Periodo} \\ \ \\
				\begin{tabular}{||l|c|c|c|c|c||}
					\hline \hline
					& Lunes & Martes  & Miércoles & Jueves  & Viernes \\ \hline \hline
					15:30 - 16:30  &       & Aula 12 &           & Aula 12 &         \\ \hline
					16:30 - 17:30  &       & Aula 12 &           & Aula 12 &         \\ \hline
					17:30 - 18:30  &       &         &           &         &         \\ \hline
					18:30 - 19:30  &       &         &           &         &         \\ \hline
					\hline
				\end{tabular}
			\end{center}
		}
		\only<0|handout:1>{
			\begin{center}
				\textbf{Tercer (3) y Cuarto (4) Periodo} \\ \ \\
				\begin{tabular}{||l|c|c|c|c|c||}
					\hline \hline
					& Lunes & Martes  & Miércoles & Jueves      & Viernes     \\ \hline \hline
					15:30 - 16:30  &       & Aula 12 &           & Aula 12 (4) &             \\ \hline
					16:30 - 17:30  &       & Aula 12 &           & Aula 12 (4) &             \\ \hline
					17:30 - 18:30  &       &         &           &             & Aula 12 (3) \\ \hline
					18:30 - 19:30  &       &         &           &             & Aula 12 (3) \\ \hline
					\hline
				\end{tabular}
			\end{center}
		}
	\end{small}
	\begin{itemize}
		\item<3-> Las clases serán indistintamente de teoría o laboratorio, según convenga al ritmo de la asignatura.
	\end{itemize}
\end{frame}

\begin{frame}[c]
	\frametitle{Horario Tutorías}
	\begin{itemize}[<+->]
		\item A cualquier hora, preferentemente de mañana.
		\item Si se quiere asegurar disponibilidad, avisad con antelación.
		\item Se permiten tutorías tanto presenciales como telemáticas.
		\item Disponible a través de correo, moodle y \alert{Discord}.
	\end{itemize}
\end{frame}



\section{Métodos de Evaluación y Calificación}

\subsection{Fórmula Calificación}

\begin{frame}[c]
	\frametitle{Cálculo de la Calificación Final}
	\begin{block}{Fórmula de Cálculo de la Calificación Final}
		\begin{center}
        \begin{tabular}{ll}
			Calificación  Final  =  & Proyecto $\times$ 0.75 + \\
                                    & Prueba  Final  Escrita $\times$ 0.25 \\
		\end{tabular}
        \end{center}
	\end{block}
	\begin{itemize}
		\item<2-> Para poder aplicar la fórmula anterior se debe obtener un mínimo de 5.00 en el proyecto.
	\end{itemize}
\end{frame}

\subsection{Ponderación de las Etapas del Proyecto}

\begin{frame}[c]
    \frametitle{Ponderación de las Etapas del Proyecto}
    \begin{center}
	\begin{tabular}{|ll|r|}
    \hline \hline
	E01 & Modelo de Domino     & 10\% \\ \hline
    E02 & Domain-Driven Design & 10\% \\ \hline	
    E03 & Implementación POJOs &  5\% \\ \hline
    E04 & Anotaciones JPA                  & 10\% \\ \hline
    E05 & Desarrollo de los Repositorios   &  5\% \\ \hline
    E06 & Especificación de la API REST    & 10\% \\ \hline
    E07 & Implementación del Controlador HTTP   &   10\% \\ \hline
    E08 & Implementación de la Capa de Servicio &  10\% \\ \hline
    E09 & Implementación del Router SPA         & 10\% \\ \hline
    E10 & Implementación de Compomentes         & 15\% \\ \hline
    E11 & Implementación de LLamadas AJAX       &  5\% \\ \hline
    E12 & Implementación del Control de Accesos & 10\% \\
    \hline \hline
	\end{tabular}
    \end{center}
\end{frame}

\subsection{Prueba Final}

\begin{frame}[c]
    \frametitle{Prueba Final}
	\begin{enumerate}[<+->]
		\item La prueba contendrá ejercicios y cuestiones sobre razonamientos teóricos.
        \item Se podrá hacer uso de todo tipo de material escrito.
        \item En ningún caso se dejará hacer uso de de dispositivos electrónicos, especialmente, de aquellos con capacidades de comunicación inalámbrica.
		\item \alert{El material escrito debe servir para consultar cuestiones puntuales, pero en el caso ideal no debería hacerse ningún uso de los mismos}.
		\item Aquellas respuestas que sean una simple copia de lo disponible en algún tipo de material de la asignatura se calificarán con 0 puntos.
	\end{enumerate}
\end{frame}

\section{Bibliografía}

\begin{frame}[c]
	\frametitle{Bibliografía Principal}
    \begin{thebibliography}{1}

\bibitem{Fowler2002}
Martin Fowler.
\newblock {\em {Patterns of Enterprise Application Architecture}}.
\newblock Addison-Wesley Professional, 2002.

\bibitem{Evans2002}
Eric J. Evans.
\newblock {\em {Domain-Driven Design}}.
\newblock Pearson, 2003.

\bibitem{Ricahrdson2007}
Leonard Richardson and Sam Ruby.
\newblock {\em {RESTful Web Services}}.
\newblock  O'Reilly, 2007.

\bibitem{}
Emmit Scott.
\newblock {\em {SPA Design and Architecture: Understanding Single Page Web Applications}}.
\newblock  Manning Publications, 2015.

\end{thebibliography}
\end{frame}

\end{document} 